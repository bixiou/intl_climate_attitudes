\section{United States\label{app:US}}

% \subsection{Knowledge about climate change}

% In France most people believe that climate change is anthropogenic. 57\% of them believe that human activity causes ``a lot'' or ``most'' of climate change (see Panel A Figure \ref{fig:app_US_knowledge_desc}). Yet, France falls below the 70\% average in high-income countries along this dimension (see Figure \ref{fig:knowledge_positive}). Furthermore, around one half of respondents believe that cutting GHG emissions by half would be sufficient to stop the rise in temperatures (see Panel B Figure \ref{fig:app_US_knowledge_desc}). 

% Only around half of all respondents are able to correctly rank modes of transportation and electricity production by emissions. 
% Regarding \textit{per capita} footprint only a minority knows the full correct ranking (see Panel C Figure \ref{fig:app_US_knowledge_desc}). Respondents understand that sea-level rises and severe droughts and heatwaves are more likely if nothing is done to limit climate change, but at the same time only 37\% know that it is unlikely that it will lead to more frequent volcanic eruptions (see Panel D Figure \ref{fig:app_US_knowledge_desc}).

% As shown in Figure \ref{fig:app_US_knowledge_setA}, there are some differences across respondents regarding their knowledge about climate change. Most notably, people with a college degree and left-leaning or center voters are more accurate than right-leaning voters and very left-leaning ones. 
% Age and income are not significantly correlated with knowledge. Overall, the patterns in terms of demographic and energy characteristics for France are similar than when considering all countries together. 

% \subsection{Who supports climate action?}

% From Figure \ref{fig:app_US_support_desc}, it appears that in France, most people (57\%) support a green infrastructure program, 25\% are indifferent and 17\% oppose it. Bans on combustion engines and a carbon tax with cash transfers only have support from around 30\% of respondents (against 43\% on average in high-income countries, see Figure \ref{fig:national_policies}). There is strong opposition to taxes on fossil fuel and on cattle-related products. Lastly, there is a clear majority that supports subsidies to low-carbon technologies, the mandatory and subsidized insulation of buildings, a ban of polluting cars in city centers, a carbon tax whose revenue is earmarked (except if it is for a reduction in public deficit, equal cash transfers to all households or a reduction in corporate income taxes), subsidies or organic and local vegetables, and the ban of intensive cattle farming.

% In Figure \ref{fig:app_US_willingness_desc}, we see that in France people are most willing to limit flying and switching to an electric vehicle, and are less willing to limit driving or limit their beef consumption (even though a minority indicates being a little or not at all willing to adopt those behaviors). The most important condition for them to adopt those behaviors is that the most well-off also change their behaviors.

% In France, age has a negative effect on support. People over 35 years old support less the main climate policies compared to people aged between 18 and 24 years old. Important predictors are also the economic leaning, dependency on a car, and the frequency of beef consumption (see Figure \ref{fig:app_US_support_setAB}).

% \subsection{Perceived effects of climate policies}

% Figure \ref{fig:app_US_pol_prop} shows that in France, 67\% of people somewhat or strongly agree that a green infrastructure program would reduce air pollution. 40\% of people think that high-income earners would mostly win or win a lot from a green infrastructure program, and 57\% somewhat support or strongly support it. There is a majority that believes that the main climate policies would reduce air pollution. However, only 43\% believe that a carbon tax with cash transfers would encourage people to drive less, while 54\% think that a green infrastructure program would increase the use of public transport. Overall, people tend to think that the main climate policies would be effective to fight climate change, but that it will be costly (except for a ban of combustion-engine cars) and that they would not have positive economic effects. They also tend to believe that the distributional impacts would be negative.

% \subsection{What reasoning underlies support for climate polices?}

% In France, similarly to the rest of the countries surveyed, reasoning about the effects of the main climate policies is what matters most for support. In particular, believing that a policy would reduce emissions is the most important reasoning. Reasoning about distributional impacts matters more for the support of a carbon tax with cash transfers compared to the two other policies. There is a similar pattern about believing that inequality is an important problem (see Figure \ref{fig:app_US_support_setC}).

% In France, watching both treatment videos is what usually increases support the most (see Figure \ref{fig:app_US_suppot_setAt}). The effect of the \textit{climate impacts} video is not usually significant, while the \textit{climate policies} video increases support for a ban on combustion-engine cars and a carbon tax with cash transfers (i.e., the policies on which the video focuses).

% \subsection{Latent Dirichlet Allocation}
% We apply an unsupervised, clustering machine learning algorithm based on a latent Dirichlet allocation (LDA) to identify profiles of respondents. The algorithm classifies respondents in profiles according to their preferences and reasoning about climate policies. The algorithm identifies four different profiles of respondents. The share of each profile in the sample is reported in brackets, and the distribution of profiles across different socio-demographic and energy usage characteristics is shown in Figure \ref{fig:app_US_LDA_desc}.
% \begin{itemize}
%     \item \textbf{Efficiency-focused supporters} [31\%]: They typically support the main climate policies (a green infrastructure program, a ban on combustion-engine cars, a carbon tax with cash transfers), as well as a carbon tax if the revenue is earmarked to fight climate change (e.g., for funding environmental infrastructures or subsidizing low-carbon technologies). They also support a broad set of other climate policies (e.g., a ban on polluting cars in city centers, funding clean energy in low-income countries, a tax on flying, a ban on combustion-engine cars with alternatives available, subsidies on organic and local vegetables, subsidies to low-carbon technologies), including some of the policies with the less overall support (e.g., a ban of intensive cattle farming, a tax on fossil fuels, the removal of subsidies for cattle farming, or a high tax on cattle products). Moreover, their reasoning is usually based on efficiency considerations: they believe that the main climate policies would reduce pollution and emissions. They do not believe that their own household or that high-income earners would lose from those policies. They worry about the consequences of climate change, trust the government, and believe inequality is an important problem. They understand emissions across activities and regions.
%     \item \textbf{Compensation-focused supporters} [17\%]: They typically support a carbon tax if its revenue is used as a transfer. In particular, they support a carbon tax if the revenue is used to finance a reduction in personal income taxes, cash transfers to households with no alternative to using fossil fuels, cash transfers to the poorest households, tax rebates for the most affected firms, or a reduction in the public deficit. They also support subsidies to low-carbon technology. While they know that climate change is real and caused by humans, they do not understands its impacts, nor know which gases cause it. Their reasoning is based on distributional effects: they believe low-income earners would lose from the main climate policies, that high-income earners would win, and that inequality is an important problem. They also believe that they will suffer from climate change, that net-zero is technically feasible, worry about the consequences of climate change, and believe the main climate policies would have positive economic effects.
%     \item \textbf{Indifferent to climate policies} [23\%]: They typically neither support nor oppose climate policies. They are indifferent to the main climate policies (a green infrastructure program, a ban on combustion-engine cars, a carbon tax with cash transfers). They are also indifferent to a broad set of other climate policies (e.g., mandatory and subsidized insulation of buildings, funding clean energy in low-income countries, the removal of subsidies for cattle farming). Using the revenue of a carbon tax for different purposes do not make them less indifferent (e.g., they are indifferent to a carbon whose revenue is used to finance a reduction in the public deficit, a reduction in corporate income taxes, subsidies to low-carbon technologies, tax rebates for the most affected firms). They usually believe that high-income earners would win from the main climate policies and that inequality is not an important problem. They do not believe the main climate policies would be effective (i.e., that they would reduce pollution and emissions, nor that they would have positive economic effects). They do not worry about the consequences of climate change, do not understands emissions across activities and regions, and do not understand the impacts of climate change, but do know which gases cause climate change.
%     \item \textbf{Opponents of climate policies} [29\%]: They typically oppose climate policies. In particular they oppose taxes (e.g., a tax on fossil fuels, a high tax on cattle products, a carbon tax with cash transfers, a tax on flying), but their opposition also applies to a broader set of policies (e.g., a ban on combustion-engine cars, the removal of subsidies for cattle farming, a ban on combustion-engine cars with alternatives available, a ban of cars in city centers). Their reasoning is based on self-interested and distributional effects: they believe their own household and low-income earners would lose from the main climate policies. They do not believe net-zero is technically feasible and do not trust the government. Moreover, they usually do not think the main climate policies are efficient (i.e., that they would reduce emissions or pollution, and would have positive economic effects).
% \end{itemize}
\clearpage

\begin{figure}[h!]
    \caption{Knowledge about climate change}\label{fig:app_US_knowledge_desc}
    \begin{center}
        \begin{tabular}{cc}
            \textsc{Panel A: ``What part of climate change} & \textsc{Panel B: ``Do you think that} \\
            \textsc{do you think is due to human activity?''} & \textsc{cutting global GHG emissions by half} \\
             & \textsc{would be sufficient to eventually}\\
              & \textsc{stop temperatures from rising?''}\\
			  \includegraphics[width=2.5in]{figures/appendix/US/CC_anthropogenic.pdf} & \includegraphics[width=2.5in]{figures/appendix/US/CC_dynamic.pdf}
			  %\includegraphics[width=2.5in]{figures/appendix/US/Appendix_Fig1A_US.pdf} & \includegraphics[width=2.5in]{figures/appendix/US/Appendix_Fig1B_US.pdf}
        \end{tabular}
        \begin{tabular}{c}
            \textsc{Panel C: GHG Emission Ranking} \\
            %\includegraphics[width=6in]{figures/appendix/US/Appendix_Fig1C_US.pdf} 
			\includegraphics[width=6in]{figures/appendix/US/CC_impacts3.pdf} \\
            % \textsc{Panel D: Climate Change Gases} \\
            % \includegraphics[width=4.5in]{figures/appendix/US/Appendix_Fig1D_US.pdf} \\
            \\
            \textsc{Panel D: ``If nothing is done to limit climate change,}\\
            \textsc{how likely do you think it is that climate change will}\\
            \textsc{lead to the following events?''} \\
            %\includegraphics[width=4in]{figures/appendix/US/Appendix_Fig1E_US.pdf}
			\includegraphics[width=4in]{figures/appendix/US/footprint2.pdf}
        \end{tabular}
    \end{center}
\end{figure}


\begin{figure}[h!]
    %\centering
    \caption{Who has better knowledge about climate change?}\label{fig:app_US_knowledge_setA}
    %Knowledge of climate change regressed on individual characteristics.}
    \vspace{-.5em}
    \hspace{-5em}
    \begin{center}
    \makebox[\textwidth][c]{\includegraphics[width=\textwidth, right]{figures/appendix/US/Coefplot_SetA_iknow_all_US.pdf}} \label{fig:Coefplot_SetA_iknow_all_US}
    \end{center}
\vspace{.1cm}
\par
 {\footnotesize \textit{Note}: The figure shows the coefficients from an OLS regression of the \textit{Knowledge index} on socio-economic indicators (left panel) and on socio-economic and energy usage indicator (right panel). Results are displayed both for the entire sample (in blue) and for the French sample (in red). All covariates are indicator variables. Treatment indicators (as well as country fixed effects for the entire sample) are included but not displayed. See Appendix \ref{app:variables} for variable definitions and notes under Figure \ref{fig:knowledge_reg} for a list of the omitted categories.}
\end{figure}


\begin{figure}[h!]
    %\centering
    \caption{Share of respondents who support or oppose climate change policies.}\label{fig:app_US_support_desc}
    \begin{center}
    \makebox[\textwidth][c]{\includegraphics[width=\textwidth]{figures/appendix/US/national_policies.pdf}} %\label{fig:national_policies}
    \end{center}
    
    {\footnotesize \textit{Note}: Opposition or support is asked on a 5-point scale with ``Indifferent'' as the middle option. For the formulation of each question, see Appendix \ref{app:questionnaire}.}
\end{figure}

\begin{figure}[h!]
    %\centering
    \caption{%Share of people willing ``A lot'' or ``A great deal'' to adopt different behaviors.
    Share of people's answers to the willingness to adopt different behaviors and to the importance of different conditions for such adoption.}\label{fig:app_US_willingness_desc}
    \makebox[\textwidth][c]{\includegraphics[width=\textwidth]{figures/appendix/US/willingness_conditions_all.pdf}} % willing_positive_countries willing_all_positive_countries
    %\label{fig:willing}
    
    {\footnotesize \textit{Note}: \textit{Behaviors} relate to the question ``\textit{To what extent would you be willing to adopt the following behaviors?}'' and \textit{Factors} to ``\textit{How important are the factors below in order for you to adopt a sustainable lifestyle (i.e. limit driving, flying, and consumption, cycle more, etc.)?}''. Both questions use a 5-point scale.
    %Real-stake questions include the signature of a petition to ``stand up for real climate action'' and an indicator that the respondent chose to  donate more than 20\% of the prize of about \$100 should they win the survey's lottery.
    }   
\end{figure}

\begin{figure}%[h!]
    %\centering
    \caption{Which respondents support climate action? \label{fig:app_US_support_setAB}}
    \vspace{-.5em}
    \hspace{-5em}
    \begin{center}
    \makebox[\textwidth][c]{\includegraphics[width=\textwidth]{figures/appendix/US/Coefplot_SetAB_imainpols_all_US.pdf}}     \label{fig:Coefplot_SetAB_imainpols_all_US}
    \end{center}

 {\footnotesize \textit{Note}: The figure shows the coefficients from an OLS regression of the \textit{Knowledge index} on socio-economic indicators (left panel) and on socio-economic and energy usage indicator (right panel). Results are displayed both for the entire sample (in blue) and for the French sample (in red). All covariates are indicator variables. Treatment indicators (as well as country fixed effects for the entire sample) are included but not displayed. See Appendix \ref{app:variables} for variable definitions and notes under Figure \ref{fig:support} for a list of the omitted categories.}
\end{figure}

\begin{figure}[h!]
    %\centering
    \caption{Properties of the main policies. [Share of agreement]\label{fig:app_US_pol_prop}}
    \begin{center}
    \makebox[\textwidth][c]{\includegraphics[width=\textwidth]{figures/appendix/US/Heatplot_main_policies_all_win_positive_US.pdf}} %\label{fig:Heatplot_main_policies_all_win_positive_countries}
    \end{center}
    
    {\footnotesize \textit{Note}: Questions use 5-point scales: \textit{Disagree/Agree} for Effects and Is fair, \textit{Lose/Win} for Distributional impacts, and \textit{Oppose/Support} for Support.
    %For results excluding the middle answers ``\textit{Neither agree or disagree}'', ``\textit{Neither win nor lose}'' or ``\textit{Indifferent}'', see Figure \ref{fig:Heatplot_main_policies_all_win_share_countries}.
    For the formulation of each question, see Appendix \ref{app:questionnaire}.}
\end{figure}

\begin{figure}[h!]
    %\centering
    \caption{Reasoning underlying support for main climate policies.% \\ Effects of beliefs in the simple OLS regression of (normalized) support to main policies on individual characteristics and beliefs.
    \label{fig:app_US_support_setC}}
    \begin{center}
    \makebox[\textwidth][c]{\includegraphics[width=\textwidth]{figures/appendix/US/Coefplot_SetAC_PlotCdis_combined_pols_US.pdf}} % Coefplot_SetAC_PlotC_index_main_policies_all
    \label{fig:Coefplot_SetAC_PlotCdis_combined_pols_US}
    \end{center}
    
 {\footnotesize \textit{Note}: The figure shows the coefficients from a regression of support for each policy (indicator variable) on variables measuring respondents' beliefs and perceptions, all standardized. Treatment and individual socio-demographic characteristics are included but not displayed. See Appendix \ref{app:variables} for variable definitions. A coefficient corresponds to the percentage point increase in Support following a one standard deviation increase in the covariate.}
\end{figure}

\begin{figure}[h!]
    %\centering
    \caption{%Treatment effects (of watching informational videos) in regressions of support for the main policies on the set (A) of socio-demographics. All variables other than the treatment are set at their sample mean.
    Effects of the Treatment on Support for Climate Action
    \label{fig:app_US_suppot_setAt}}
    \begin{center}
    \makebox[\textwidth][c]{\includegraphics[width=\textwidth]{figures/appendix/US/Coefplot_SetAT_PlotT_support_main_others_US.pdf}} \label{fig:Coefplot_SetAT_PlotT_support_main_others_US} 
    \end{center}
    
     {\footnotesize \textit{Note}: The figure shows the coefficients from a regression of the indicator variables listed on the left, capturing support for various policies and willingness to change behaviors, on indicators for each treatment, controlling for socio-economic characteristics, i.e., age, gender, income, education, economic leaning, living with a child (controls are not displayed). See Appendix \ref{app:variables} for variable definitions.}
\end{figure}

\begin{figure}[h!]
    %\centering
    \caption{Distributions of profiles by socio-demographic and energy usage characteristics \label{fig:app_US_LDA_desc}}
    \makebox[\textwidth][c]{\includegraphics[width=\textwidth]{figures/appendix/US/topic_factor_subgroups_US.pdf}} %\label{fig:lda}
    
    {\footnotesize \textit{Note}: This figure corresponds to the distribution of the LDA profiles across different socio-demographic and energy usage characteristics.}
\end{figure}